%!TEX root = ../Main.tex

%Define the document%
\documentclass[12pt, a4paper]{article}
% \documentclass[12pt, a4paper, draft]{article}

% \usepackage[utf8]{inputenc} % LaTeX Fonts only
\usepackage[T1]{fontenc}  % XeLaTeX Fonts -- Change fonts in the following
\usepackage{mlmodern}     % XeLaTeX Fonts -- Modern Latin Modern (blacker)
% \usepackage{lmodern}      % LaTeX Fonts 
%% Use the following for system fonts
% \usepackage{fontspec} % Enable this to change to system fonts
% \setmainfont{Arial}   % System Font ie Arial

\usepackage[ddmmyyyy]{datetime}
\usepackage[margin=2cm]{geometry}

\usepackage{amsmath, amsfonts, amssymb}
\usepackage{graphicx}
\usepackage{float}
\usepackage{placeins} % Float barrier

\usepackage{enumitem}

\usepackage[table]{xcolor}
\usepackage{multirow}     % Allow multi-row and multicolumn, table modification
\usepackage{xcolor}       % Include Colours
\usepackage[framemethod=tikz]{mdframed} % Frames and Borders
\usepackage{subcaption}   % caption 4A,4B etc.

\usepackage{listings}     % Code listings
\usepackage{matlab-prettifier} % Code listings


% If its a draft, make it easier to markup the printed document
\usepackage{ifdraft}
\ifdraft{\usepackage[none]{hyphenat}}{}
\ifdraft{\usepackage{color, soul}}{}
\ifdraft{\usepackage{setspace}}{}
\ifdraft{\doublespace}{}


% The following packages are needed for 
% some matlab2tikz features, otherwise can be ignored.
\usepackage{pgfplots}
\usepackage{grffile}
\pgfplotsset{compat=newest}
\usetikzlibrary{plotmarks}
\usetikzlibrary{arrows.meta}
\usepgfplotslibrary{patchplots}


% Headers, Footers
\usepackage{fancyhdr}     % Fancy Header Footers
\pagestyle{fancy}
\fancyhf{}
\fancyhfoffset[L]{2em}    % left extra length rule
\fancyhfoffset[R]{2em}    % right extra length rule
\renewcommand{\headrulewidth}{1.2pt}
\renewcommand{\footrulewidth}{0.4pt}
\fancyhead{}
\setlength{\headheight}{15pt}
\fancyhead[C]{\textbf{\headerTitle}}
\fancyfoot{}
\fancyfoot[C]{\thepage}

\usepackage{natbib} %% Do not use natbib, use bibtex/biber - this bst has been modified extensively to suit.
\bibpunct{(}{)}{;}{a}{}{,} %Sets the punctuation for UniSA Style
% \bibliographystyle{/Users/kobrien/Documents/Study/unisa2020}
\bibliographystyle{sections/fixed/unisa2020}

% Control the indent; Kane strongly dislikes indents.
\setlength{\parindent}{0em}
% Rename "Contents"; Contents is ugly.
\renewcommand{\contentsname}{Table Of Contents} % Default "Contents"
\numberwithin{equation}{section}

% Set some common math letters:
\newcommand{\R}{$\mathbb{R}$}
\newcommand{\I}{$\mathbb{I}$}
\newcommand{\C}{$\mathbb{C}$}
\newcommand{\Z}{$\mathbb{Z}$}
\newcommand{\ihat}{\textbf{\^\i}}
\newcommand{\jhat}{\textbf{\^\j}}
\newcommand{\khat}{\textbf{\^k}}

% Set some common abbreviations:
\newcommand{\ie}{\textit{i}.\textit{e}.,}
\newcommand{\eg}{\textit{e}.\textit{g}.,}

\usepackage{hyperref}
% \usepackage{bookmark}
\hypersetup{
    colorlinks=true,
    linkcolor=blue,
    citecolor=black,
    filecolor=magenta,      
    urlcolor=magenta,
    pdftitle={\submissionTitle},
    pdfauthor={Kane O'Brien}
    pdfpagemode=FullScreen,
    bookmarksopen=true
    }

%% Past this is relatively unimportant packages.

% Create new environments (Question, Answer, Code, Example, Titles)
% using colour blocks 
% The colour /style is defined After this
\newmdenv[style=bluestyle]{quest}
\newenvironment{question}[1]
  {\begin{quest}[frametitle=#1]}
  {\end{quest}}

\newmdenv[style=redstyle]{ans}
\newenvironment{answer}[1]
  {\begin{ans}[frametitle=#1]}
  {\end{ans}}

\newmdenv[style=greenstyle]{code}
\newenvironment{codeblock}[1]
  {\begin{code}[frametitle=#1]}
  {\end{code}}

\newmdenv[style=graystyle]{ex}
\newenvironment{example}[1]
  {\begin{ex}[frametitle=#1]}
  {\end{ex}}

\newmdenv[style=greenNoTitle]{titles}
\newenvironment{title1}[1]
  {\begin{titles}}
  {\end{titles}}

% Define the colours and environment for block styles.%
%Blue Section Blocks
\mdfdefinestyle{bluestyle}{%
linecolor=blue!100!,outerlinewidth=1pt,%
frametitlerule=true,frametitlefont=\sffamily\bfseries\color{white},%
frametitlerulewidth=1pt,frametitlerulecolor=blue!100,%
frametitlebackgroundcolor=blue!100!,
backgroundcolor=blue!5!,
innertopmargin=\topskip,
roundcorner=4pt
}

%Red Section Blocks
\mdfdefinestyle{redstyle}{%
linecolor=red!100,outerlinewidth=1pt,%
frametitlerule=true,frametitlefont=\sffamily\bfseries\color{white},%
frametitlerulewidth=1pt,frametitlerulecolor=red!100,%
frametitlebackgroundcolor=red!100,
backgroundcolor=red!10!,
innertopmargin=\topskip,
roundcorner=4pt
}

%Green Section Blocks
\mdfdefinestyle{greenstyle}{%
linecolor=green!60!black,outerlinewidth=1pt,%
frametitlerule=true,frametitlefont=\sffamily\bfseries\color{white},%
frametitlerulewidth=1pt,frametitlerulecolor=green!60!black,%
frametitlebackgroundcolor=green!60!black,
backgroundcolor=green!10!,
innertopmargin=\topskip,
roundcorner=4pt
}

%Grey Section Blocks
\mdfdefinestyle{graystyle}{%
linecolor=gray,outerlinewidth=1pt,%
frametitlerule=true,frametitlefont=\sffamily\bfseries\color{white},%
frametitlerulewidth=1pt,frametitlerulecolor=gray,%
frametitlebackgroundcolor=gray,
backgroundcolor=lightgray!20,
innertopmargin=\topskip,
roundcorner=4pt
}

\mdfdefinestyle{greenNoTitle}{%
linecolor=green!60!black,outerlinewidth=1pt,%
frametitlerule=false,frametitlefont=\sffamily\bfseries\color{white},%
frametitlerulewidth=0pt,frametitlerulecolor=green!60!black,%
frametitlebackgroundcolor=green!60!black,
backgroundcolor=green!10,
% innertopmargin=\topskip,
roundcorner=4pt
}
